% This text is proprietary.
% It's a part of presentation made by myself.
% It may not used commercial.
% The noncommercial use such as private and study is free
% Sep. 2005 
% Author: Sascha Frank 
% University Freiburg 
% www.informatik.uni-freiburg.de/~frank/


\documentclass{beamer}
%\usepackage[spanish]{babel}
\usepackage[utf8]{inputenc}
\usetheme{Antibes}
%\selectlanguage{spanish}
\begin{document}
\title{Introducción a programación en Android}   
\author{Alexandre Paz, erzapito@gmail.com} 
\date{\today} 

\frame{\titlepage} 

\frame{\frametitle{Table of contents}\tableofcontents} 

\section{Datos generales}
\frame{\frametitle{Información general}
\begin{itemize}
 \item Se programa en java
 \item Mucha información se añade en XML
 \item Permite incluir librerías java y librerías Android
 \item Permite añadir datos extra en el propio empaquetado
 \item Limitaciones en tamaño, número de métodos.
 \item Para mayor rendimiento se puede programar con C++ (NDK) y cargar las librerías usando JNI.
\end{itemize}
}

\frame{
\frametitle{Componentes principales}
\begin{itemize}
 \item AndroidManifest.xml: todos los componentes de la aplicación
 \item Activities: procesos visibles por el usuario
 \item Services: procesos en segundo plano
 \item Receivers: recepción de eventos
\end{itemize}
}

\section{Interfaz}
\frame{
\frametitle{Actividades}
\begin{itemize}
 \item Son interacciones únicas con el usuario, como listar elementos, ver un elemento, etc...
 \item Contienen fragmentos y vistas
 \item Los fragmentos son conjuntos de vistas que pueden ser cargados por diferentes actividades
\end{itemize}
}

\frame{
\frametitle{Vistas}
\begin{itemize}
 \item Etiquetas
 \item Botones
 \item Cajas de texto
 \item Imágenes
 \item Listados
 \item Layouts
 \item ...
\end{itemize}
}

\frame{
\frametitle{Layouts}
Vistas especiales que agrupan varias vistas y definen como se colocan
\begin{itemize}
 \item LinearLayout
 \item FrameLayout
 \item RelativeLayout
\end{itemize}
}

\frame{
\frametitle{Recursos}
Pueden ser ajustados en base a distintos parámetros como el idioma, país, densidad de pantalla, tamaño de pantalla, etc...

\begin{itemize}
 \item Imágenes
 \item Layouts
 \item Menus
 \item Textos
 \item Dimensiones
 \item Estilos
 \item Animaciones
\end{itemize}
}

\frame{
\frametitle{Animaciones}
Permiten animar vistas.
\begin{itemize}
 \item animaciones por XML: aplicacables a cualquier vista
 \item Animations: aplicables a cualquier vista
 \item Animator: aplicables a cualquier cosa
\end{itemize}
}

\frame{
\frametitle{Dialogos}
\begin{itemize}
 \item Builders
 \item Fragmentos
 \item Predefinidos
\end{itemize}
}

\section{Trabajos en segundo plano}
\frame{
\frametitle{Servicios}
\begin{itemize}
 \item Tareas de larga duración a ejecutar en segundo plano.
 \item Pueden comunicarse con las actividades para enviar información.
 \item Para realizar trabajos siguen necesitando hilos o equivalentes.
\end{itemize}
}

\frame{
\frametitle{Hilos de trabajo}
\begin{itemize}
 \item Hilo de interfaz: se encarga de pintar las vistas, animaciones y eventos de usuario. Soltará una excepción si se intentan comunicaciones de red.
 \item Hilos secundarios: creados a mano o a través de otros componentes, no pueden modificar las vistas.
\end{itemize}
}

\frame{
\frametitle{Componentes para trabajar con hilos}
\begin{itemize}
 \item Threads: segundo plano
 \item Handlers: hilo de interfaz
 \item AsyncTask: mezcla de segundo plano e interfaz
\end{itemize}

}

\frame{
\frametitle{AsyncTask}
\begin{itemize}
 \item la tarea principal se ejecuta en segundo plano
 \item tiene metodos de ayuda para el hilo de interfaz para antes, despues y en proceso de la tarea principal.
 \item facilita transferir información del hilo de segundo plano al de interfaz
\end{itemize}
}


\section{Persistencia}

\frame{
\frametitle{Opciones}
\begin{itemize}
 \item Ficheros: privados y publicos
 \item Base de datos: sqlite3
 \item Preferencias: mapas de nombre a valor escalar
\end{itemize}
}

\frame{
\frametitle{Base de datos}
\begin{itemize}
 \item Hacen falta métodos para crear y actualizar
 \item Existen librerías para facilitar el acceso, como ActiveAndroid
\end{itemize}
}

\section{Comunicación entre procesos}

\frame{
\frametitle{Intent}
Permiten solicitar al sistema ejecutar una acción.
\begin{itemize}
 \item Acción Genérica: SHARE para compartir algo
 \item Acción Concreta: lanzar una actividad de la aplicación.
 \item Launcher: para que aparezcan en el menu de aplicaciones.
 \item Broadcast: eventos para notificar.
 \item permiten añadir parámetros escalares.
\end{itemize}
}

\frame{
\frametitle{Intent}
Permiten añadir flags para controlar la ejecución de actividades.
\begin{itemize}
 \item Eliminar las actividades anteriores
 \item Lanzar una nueva actividad sin reutilizar la anterior
 \item Lanzar una nueva actividad reutilizando la anterior
 \item Lanzar una actividad en un flujo de trabajo diferente.
\end{itemize}
}

\section{Testeo}

\frame{
\frametitle{}
\begin{itemize}
 \item Incluye una extensión de JUnit3 para pruebas unitarias. Limitadas por ciertas condiciones, como dialogos modales.
 \item Incluye simuladores de dispositivos, aunque se pueden ejecutar todo en dispositivos reales.
 \item Existen librerías que permiten probar sobre una máquina virtual común, como Robolectric.
\end{itemize}

}

\section{Conclusión}
\frame{
https://github.com/erzapito/EE23-Taller-Android
}

\end{document}

